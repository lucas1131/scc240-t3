\documentclass[12pt,a4paper]{article}

\usepackage[top=2.5cm,bottom=2cm,left=2cm,right=2cm]{geometry} 
\usepackage[brazil]{babel}
\usepackage{blindtext}
\usepackage{ragged2e}
\usepackage{soul}
\usepackage[utf8]{inputenc}
\usepackage{indentfirst}
\usepackage{mathtools}
\setlength\parindent{24pt}

% for counting comma separated words
\usepackage{expl3}
\usepackage{xparse}

\ExplSyntaxOn
% Dont use this
\NewDocumentCommand \countItemsInDefinedList { m } {
    \clist_count:N #1
}

% Use this
\NewDocumentCommand \countitems { m } {
    \clist_count:n {#1}
}
\ExplSyntaxOff

%capa
\usepackage{multicol}
\usepackage{multirow}
\usepackage{graphicx}
\usepackage{float}
\usepackage{setspace} % espaçamento
\usepackage{tabularx}
\usepackage{booktabs}
\usepackage{array}

\newcommand{\university}{Universidade de São Paulo - ICMC}
\newcommand{\discipline}{SCC0540 - Bases de Dados}
\newcommand{\data}{$1^o$ semestre / 2017}
\newcommand{\teacher}{Prof. Robson Leonardo Ferreira Cordeiro}
\newcommand{\PAE}{PAEs: Gabriel Spadon de Souza, Paulo Henrique de Oliveira}
\newcommand{\specification}{Prática 2: Modelo Relacional}

% Create relation
% #1 - Relation name
% #2 - Attributes
\newcommand{\createrelation}[2]{#1(#2) \hspace{0.3cm} N = \countitems{#2}}

% Create domain
% #1 - Domain name
% #2 - Attribute
% #3 - Domain definition
\newcommand{\createdomain}[3]{
    $Dom_{#1}$(#2) = {\bf #3}
}

% Primary & Secundary key  & Null attribute- attribute
\newcommand{\primarykey}[1]{\ul{#1}}
\newcommand{\secondarykey}[1]{\ul{\ul{#1}}}
\newcommand{\nullatt}[1]{#1$^*$}

% Super key
% #1 - super key name
% #2 - attributes that compose the key
\newcommand{\superkey}[2]{$SC$(#1) = {#2}}
\newcommand{\rarrow}{$\Longrightarrow$}

\newcommand{\members}{
    \begin{table}[!ht]
        \centering
        \begin{tabular}{ll}
            \large\textsc{Giovanna Oliveira Guimarães} & \large\textsc{Nº USP: 9293693} \\
            \large\textsc{José Augusto Noronha de Menezes Neto} & \large\textsc{Nº USP: 9293049} \\
            \large\textsc{Letícia Rina Sakurai} & \large\textsc{Nº USP: 9278010} \\
            \large\textsc{Lucas Alexandre Soares} & \large\textsc{Nº USP: 9293265} \\
        \end{tabular}
    \end{table}{}
}

\newcommand{\capaicmc}{
    \begin{center}
        \begin{center}
            \begin{table}[!ht]
                \centering 
                \begin{tabular}{cl}
                    \multirow{4}{*}{\includegraphics[height=1.8cm,keepaspectratio=true]{logo-header.png}}
                    & \university\\
                    %& \course\\ 
                    & \discipline\\
                    & \teacher\\
                    %& \PAE\\
                \end{tabular}
            \end{table}
        \end{center}
        
        \vfill
        
        {\huge \specification}
        
        \vfill
        
        \doublespacing
        \large\textsc{\members}
        
        \vfill
        
        \large São Carlos - SP \\
        \large \today \\

        \end{center}
        
        \newpage
}

\begin{document}

% Capa
\capaicmc

\tableofcontents
\newpage

% Seta espaçamento entre linhas
\singlespacing

% começa finalmente 
% Aqui explica como este documento está estruturado e por que ele está assim, qual o motivo de usarmos este doc
\section{Estrutura do Documento}
    Este documento explica as decisões de projeto tomadas durante a construção dos modelos conceitual e lógico. Na Sessão~\ref{sec:merx} corrigimos os problemas identificados na primeira parte do projeto além de justificar as decisões de tal etapa. Na Sessão~\ref{sec:mrel} apresentamos a parte seguinte do projeto e justificamos as decisões tomadas.


\section{Correções da Prática 1: Modelo Conceitual (MER-X)}\label{sec:merx}

    A chave da entidade Dispositivo, anteriormente identificada pelo nome do aparelho e pelo seu sistema operacional, foi substituída por um ID artificial.

    Perfil é uma entidade fraca, por isso, sua chave é composta pelo atributo "Nome de usuário", proveniente da entidade Conta e pelo seu próprio atributo "alias", o qual é, portanto, uma chave parcial e, por isso deve ser sublinhado com linha pontilhada.

    A cardinalidade da relação gerencia foi alterada de N:N para 1:N.

    Os atributos que davam aos perfis kids a possibilidade de decidirem preferências foram removidos da entidade geral Perfil e colocados na entidade Adulto. Agora somente os perfis adultos podem escolher suas preferências.

    Foram acrescentados os métodos para o cálculo dos atributos derivados: "Valor", "Período" e "Bandeira".

    A forma de pagamento foi associada à conta, para que o usuário não precise decidí-la sempre que for efetuar um pagamento. Com isso, é possível saber qual é a forma de pagamento atual de um usuário e a redundância no armazenamento foi removida.

    Foi acrescentada uma participação total na geralização da entidade Mídia. Agora toda mídia inserida deve, obrigatoriamente, ser classificada como episódio de série, ou um como um filme.

    Foram acrescentadas participações totais para garantir que não hajam episódios sem temporadas e temporadas sem série.

    O Nome da agregação Amizade foi mantido, uma vez que isto só é armazenado quando a solitação é aceita. Alterar o nome prejudicaria a semântica da modelagem, uma vez que não faz sentido uma "solicitação de amizade" recomendar algum tipo de Mídia.

    A agregação "Review pública" foi transformada em um ternário para possibilitar que um usuário possa assistir algum tipo de mídia sem ter que, obrigatóriamente, avaliá-la.

\section{Prática 2: Projeto Lógico Normalizado}\label{sec:mrel}

\subsection{Especificações dos Domínios}

    {\bf Apelido}: Texto alfanúmerico de até 20 caractéres 
    
    {\bf Banco}: Nome do banco (texto pequeno) 
    
    {\bf Bandeira}: Nome da bandeira do cartão (texto pequeno) 
    
    {\bf Big Integer}: Número inteiro que pode ter tamanho indefinido (e ser maior do que um número padrão) 
    
    {\bf Booleano}: Verdadeiro ou falso 
    
    {\bf Conta}: Número de 6 dígitos de conta bancária 
    
    {\bf CPF}: Número de 11 dígitos único  
    
    {\bf Código de Segurança}: Número inteiro de três dígitos 
    
    {\bf Data}: Representação da data do SGBD (visualização depende do local definido) 
    
    {\bf E-mail}: {\it String} única - deve seguir uma formatação padrão de e-mail como por exemplo ``s/$\wedge$[a-zA-Z0-9.\_]+@[a-zA-Z]+(\textbackslash.[a-z]+)+\$//''. 
    
    {\bf Endereço de IP}: Número de 12 dígitos descrevendo um IPv4 
    
    {\bf Faixa Etária}: Classificacao etária definida como um número inteiro positivo incluindo 0 (free for everyone) 
    
    {\bf Forma de pagamento}: cartão de crédito, paypal ou depósito bancário 
    
    {\bf Gênero}: Nome de gênero de mídia 
    
    {\bf Idade}: Número inteiro positivo 
    
    {\bf ID}: Número inteiro positivo único 
    
    {\bf Nome de arquivo}: Nome completo (caminho) do arquivo dentro do sistema 
    
    {\bf Nome}: Conjunto de nomes possíveis 
    
    {\bf Nota}: Número inteiro positivo que pode variar de 0 à 10 
    
    {\bf Número de Agência}: Número de 5 digitos 
    
    {\bf Número do cartão}: Número positivo único de 16 dígitos 
    
    {\bf Número real}: Número real 
    
    {\bf Número}: Número inteiro positivo de tamanho limitado 
    
    {\bf Senha}: Texto curto de 6 a 30 caracteres alfanuméricos e especiais (\#, \$, *, etc) 
    
    {\bf Texto curto}: Texto podendo conter números, dígitos e símbolos especiais de até 100 caractéres 
    
    {\bf Texto de preferencia}: {\it String} com o valor da configuração de preferência 
    
    {\bf Texto}: Texto comprido de tamanho variável 
    
    {\bf Usuário}: Texto alfanúmerico de até 20 caractéres permitindo apenas \_ de símbolo especial e sem espaços
    
    \singlespacing
    \subsubsection{Plano}
        \createdomain{Plano}{Nome}{Nome}

        \createdomain{Plano}{Preco}{Número real}

        \createdomain{Plano}{Qualidade}{Texto de preferencia}

        \createdomain{Plano}{Videos simultaneos}{Número}

        \createdomain{Plano}{Numero de perfis}{Número}

        \createdomain{Plano}{Outras descricoes}{Texto}


    \subsubsection{Conta Assina Plano}
        \createdomain{Conta Assina Plano}{Nome Plano}{Nome}

        \createdomain{Conta Assina Plano}{Nome de usuario Conta}{Usuário}


    \subsubsection{Assinatura}
        \createdomain{Assinatura}{Data vigor}{Data}

        \createdomain{Assinatura}{Nome Plano}{Nome}

        \createdomain{Assinatura}{Nome de usuario Conta}{Usuário}


    \subsubsection{Pagamento}
        \createdomain{Pagamento}{Codigo da fatura}{ID}

        \createdomain{Pagamento}{Data vigor Assinatura}{Data}

        \createdomain{Pagamento}{Nome Plano Assinatura}{Nome}

        \createdomain{Pagamento}{Nome de usuario Assinatura}{Usuário}

        \createdomain{Pagamento}{Valor}{Número real}

        \createdomain{Pagamento}{Periodo}{Número}


    \subsubsection{Forma de Pagamento}
        \createdomain{Forma de Pagamento}{Metodo}{Forma de pagamento}


    \subsubsection{Cartao de Credito}
        \createdomain{Cartao de Credito}{Forma de Pagamento}{Forma de Pagamento}

        \createdomain{Cartao de Credito}{Bandeira}{Bandeira}

        \createdomain{Cartao de Credito}{Codigo de seguranca}{Código de Segurança}

        \createdomain{Cartao de Credito}{Vencimento do cartao}{Data}

        \createdomain{Cartao de Credito}{Nome}{Nome}

        \createdomain{Cartao de Credito}{Numero do cartao}{Número do cartão}


    \subsubsection{Deposito Bancario}
        \createdomain{Deposito Bancario}{Forma de Pagamento}{Forma de Pagamento}

        \createdomain{Deposito Bancario}{Conta}{Conta}

        \createdomain{Deposito Bancario}{Agencia}{Número de Agência}

        \createdomain{Deposito Bancario}{Banco}{Banco}

        \createdomain{Deposito Bancario}{Nome titular}{Nome}

        \createdomain{Deposito Bancario}{Sobrenome titular}{Nome}

        \createdomain{Deposito Bancario}{CPF}{CPF}


    \subsubsection{Paypal}
        \createdomain{Paypal}{Forma de Pagamento}{Forma de Pagamento}

        \createdomain{Paypal}{Senha}{Senha}

        \createdomain{Paypal}{Email paypal}{E-mail}


    \subsubsection{Conta}
        \createdomain{Conta}{Nome de usuario}{Usuário}

        \createdomain{Conta}{Forma de Pagamento}{Forma de Pagamento}

        \createdomain{Conta}{Senha}{Senha}

        \createdomain{Conta}{Nome}{Nome}

        \createdomain{Conta}{CPF}{CPF}

        \createdomain{Conta}{Email}{E-mail}

        \createdomain{Conta}{Data de nascimento}{Data}


    \subsubsection{Perfil Prefere Genero}
        \createdomain{Perfil Prefere Genero}{Alias Perfil}{Apelido}

        \createdomain{Perfil Prefere Genero}{Nome de usuario Perfil}{Usuário}

        \createdomain{Perfil Prefere Genero}{Nome Genero}{Gênero}

        \createdomain{Perfil Prefere Genero}{Nota}{Nota}


    \subsubsection{Genero}
        \createdomain{Genero}{Nome}{Gênero}


    \subsubsection{Perfil Solicita Amizade}
        \createdomain{Perfil Solicita Amizade}{Aceitou}{Booleano}

        \createdomain{Perfil Solicita Amizade}{Data solicitacao}{Data}

        \createdomain{Perfil Solicita Amizade}{Alias fez solicitacao Perfil}{Apelido}

        \createdomain{Perfil Solicita Amizade}{Nome de usuario fez solicitacao Perfil}{Usuário}

        \createdomain{Perfil Solicita Amizade}{Alias recebe solicitacao Perfil}{Apelido}

        \createdomain{Perfil Solicita Amizade}{Nome de usuario recebe solicitacao Perfil}{Usuário}


    \subsubsection{Perfil}
        \createdomain{Perfil}{Alias}{Apelido}

        \createdomain{Perfil}{Nome de Usuario Conta}{Usuário}

        \createdomain{Perfil}{Idade}{Idade}

        \createdomain{Perfil}{Preferencia qualidade}{Texto de preferência}

        \createdomain{Perfil}{Preferencia legenda}{Texto de preferência}

        \createdomain{Perfil}{Preferencia idioma}{Texto de preferência}

        \createdomain{Perfil}{Faixa etaria}{Faixa Etária}


    \subsubsection{Perfil Assiste Midia}
        \createdomain{Perfil Assiste Midia}{Alias Perfil}{Apelido}

        \createdomain{Perfil Assiste Midia}{Nome de Usuario Perfil}{Usuário}

        \createdomain{Perfil Assiste Midia}{Titulo Midia}{Texto curto}

        \createdomain{Perfil Assiste Midia}{Id Review}{ID}

        \createdomain{Perfil Assiste Midia}{Tempo}{Número}


    \subsubsection{Review}
        \createdomain{Review}{Id}{ID}

        \createdomain{Review}{Nota}{Nota}

        \createdomain{Review}{Data nota}{Data}

        \createdomain{Review}{Comentario}{Texto}

        \createdomain{Review}{Data comentario}{Data}


    \subsubsection{Gerencia}
        \createdomain{Gerencia}{Alias adulto gerencia Perfil}{Apelido}

        \createdomain{Gerencia}{Nome de Usuario adulto gerencia Perfil}{Usuário}

        \createdomain{Gerencia}{Alias kids gerenciado Perfil}{Apelido}

        \createdomain{Gerencia}{Nome de Usuario kids gerenciado Perfil}{Usuário}


    \subsubsection{Midia}
        \createdomain{Midia}{Titulo}{Texto curto}

        \createdomain{Midia}{Tipo}{Texto curto}

        \createdomain{Midia}{Thumbnail}{Nome de arquivo}

        \createdomain{Midia}{Lancamento}{Data}

        \createdomain{Midia}{Duracao}{Número}

        \createdomain{Midia}{Sinopse}{Texto}

        \createdomain{Midia}{Classificacao}{Faixa etária}


    \subsubsection{Audio Midia}
        \createdomain{Audio Midia}{Audio}{Nome de arquivo}

        \createdomain{Audio Midia}{Titulo Midia}{Texto curto}


    \subsubsection{Legenda Midia}
        \createdomain{Legenda Midia}{Legenda}{Nome de arquivo}

        \createdomain{Legenda Midia}{Titulo Midia}{Texto curto}


    \subsubsection{Midia Pertence Genero}
        \createdomain{Midia Pertence Genero}{Titulo Midia}{Texto curto}

        \createdomain{Midia Pertence Genero}{Nome Genero}{Gênero}


    \subsubsection{Perfil Prefere Genero}
        \createdomain{Perfil Prefere Genero}{Titulo Midia}{Texto curto}

        \createdomain{Perfil Prefere Genero}{Alias Perfil}{Apelido}

        \createdomain{Perfil Prefere Genero}{Nome de Usuario Perfil}{Usuário}

        \createdomain{Perfil Prefere Genero}{Nota}{Nota}


    \subsubsection{Midia Pertence Temporada}
        \createdomain{Midia Pertence Temporada}{Titulo Midia}{Texto curto}

        \createdomain{Midia Pertence Temporada}{Numero Temporada}{Número}


    \subsubsection{Acesso}
        \createdomain{Acesso}{Timestamp}{Big Integer}

        \createdomain{Acesso}{Nome Dispositivo}{Nome}

        \createdomain{Acesso}{Alias Perfil}{Apelido}

        \createdomain{Acesso}{Ip}{Endereço de IP}


    \subsubsection{Perfil Possui Dispositivo}
        \createdomain{Perfil Possui Dispositivo}{Id Dispositivo}{ID}

        \createdomain{Perfil Possui Dispositivo}{Alias Perfil}{Apelido}

        \createdomain{Perfil Possui Dispositivo}{Nome de Usuario Perfil}{Usuário}


    \subsubsection{Dispositivo}
        \createdomain{Dispositivo}{Id}{ID}

        \createdomain{Dispositivo}{Nome}{Nome}

        \createdomain{Dispositivo}{Sistema operacional}{Nome}


    \subsubsection{Amizade}
        \createdomain{Amizade}{Id}{ID}

        \createdomain{Amizade}{Alias fez solicitacao Perfil}{Apelido}

        \createdomain{Amizade}{Nome de usuario fez solicitacao Perfil}{Usuário}

        \createdomain{Amizade}{Alias recebe solicitacao Perfil}{Apelido}

        \createdomain{Amizade}{Nome de usuario recebe solicitacao Perfil}{Usuário}


    \subsubsection{Amizade Recomenda Midia}
        \createdomain{Amizade Recomenda Midia}{Id Amizade}{ID}

        \createdomain{Amizade Recomenda Midia}{Alias fez solicitacao Perfil}{Apelido}

        \createdomain{Amizade Recomenda Midia}{Nome de usuario fez solicitacao Perfil}{Usuário}

        \createdomain{Amizade Recomenda Midia}{Alias recebe solicitacao Perfil}{Apelido}

        \createdomain{Amizade Recomenda Midia}{Nome de usuario recebe solicitacao Perfil}{Usuário}

        \createdomain{Amizade Recomenda Midia}{Titulo Midia}{Texto curto}

        \createdomain{Amizade Recomenda Midia}{Comentario de recomendacao}{Texto}


    \subsubsection{Temporada}
        \createdomain{Temporada}{Numero}{Número}

        \createdomain{Temporada}{Nome Serie}{Texto curto}

        \createdomain{Temporada}{Numero de episodios}{Número}


    \subsubsection{Serie}
        \createdomain{Serie}{Nome}{Texto curto}

        \createdomain{Serie}{Numero temporadas}{Número}

        \createdomain{Serie}{Classificacao}{Faixa etária}


    \subsubsection{Pessoa}
        \createdomain{Pessoa}{Nome}{Nome}


    \subsubsection{Pessoa Participa Midia}
        \createdomain{Pessoa Participa Midia}{Titulo Midia}{Texto curto}

        \createdomain{Pessoa Participa Midia}{Nome Pessoa}{Nome}

        \createdomain{Pessoa Participa Midia}{Diretor}{Booleano}

        \createdomain{Pessoa Participa Midia}{Ator}{Booleano}


    \subsubsection{Pessoa Participa Serie}
        \createdomain{Pessoa Participa Serie}{Nome Serie}{Texto curto}

        \createdomain{Pessoa Participa Serie}{Nome Pessoa}{Nome}

        \createdomain{Pessoa Participa Serie}{Diretor}{Booleano}

        \createdomain{Pessoa Participa Serie}{Ator}{Booleano}

\subsection{Esquema de Relações Inicial}
    
    Aqui é apresentado o esquema de relações logo após o mapeamento do MER-X para o Modelo Relacional, sem pensar em aplicar as normalizações.
    
    Nota: utilizamos a notação {\bf (fk)} para indicar que o atributo é uma chave estrangeira. O nome da chave estrangeira será sempre o nome do atributo seguido do nome da relação que ela se refere.
    
    \singlespacing

    \doublespacing
    % //(Alternativa 1_de mapeamento)
    \subsubsection{Forma de pagamento: cartão de crédito, depósito bancário e Paypal}
    
    O mapeamento foi realizado de acordo com a Alternativa 1, pois as consultas tipicamente se concentram em um ou poucos conjuntos de entidades específicas de cada vez. A especialização total é garantida com a existência uma chave estrangeira Forma de Pagamento em cada tabela específica.
        
    \createrelation{{\bf Forma de Pagamento}}{\primarykey{Metodo}}

    \createrelation{{\bf Cartao de Credito}}{\primarykey{Forma de Pagamento} (fk), Bandeira, Codigo de seguranca, Vencimento do cartao, Nome, Numero do cartao}

    \createrelation{{\bf Deposito Bancario}}{\primarykey{Forma de Pagamento} (fk), Conta, Agencia, Banco, Nome titular, Sobrenome titular, CPF}

    \createrelation{{\bf Paypal}}{\primarykey{Forma de Pagamento} (fk), Senha, Email paypal}

	\subsubsection{Mídia: filme e episódio}
    
	% //(Alternativa 2_de mapeamento)
    % //NOTE Ator só terá relacionamento com mídia caso o_Tipo seja Filme.
    A segunda alternativa foi escolhida para este mapeamento, pois os conjuntos de entidades específicas possuem poucos atributos próprios. Garantimos a especialização total colocando o atributo de classificação (tipo) como chave da tabela. Desta forma, as mídias registradas na base de dados devem pertencer, obrigatoriamente, a um dos tipos disponíveis.\\
    Contudo, utilizando esta alternativa, teremos mídias que não pertencem à temporadas (Filmes). Por isso, foi criada uma tabela para relacionar todas as mídias do tipo Episódio com suas respectivas temporadas.\\
    Para evitar redundância e, consequentemente, inconsistências, haverá relacionamento entre mídia e atores somente quando a mídia em questão for do tipo filme. Caso contrário, o ator irá relacionar-se diretamente com a série.\\
    
    % //NOTE:_Por causa do mapeamento da especialização de Midia,_temos mídias que não pertencem à_uma temporada,_então compensa mais criar uma tabela relacionando as mídias do tipo Episódio com Temporada
    
    \createrelation{{\bf Midia}}{\primarykey{Titulo}, \nullatt{Tipo}, Thumbnail, Lancamento, Duracao, Sinopse, Classificacao}

	\subsubsection{Perfil: adulto e kids}
    A segunda alternativa foi escolhida para esse mapeamento pela existência de poucos atributos próprios nos conjuntos de entidade específicos.
    
    % //(Alternativa 2_de mapeamento)
    \createrelation{{\bf Perfil}}{\primarykey{Alias}, \primarykey{Nome usuario Conta} (fk), \nullatt{Idade}, Preferencia qualidade, Preferencia legenda, Preferencia idioma, Faixa etaria}

	\createrelation{{\bf Plano}}{\primarykey{Nome}, Preco, Qualidade, Videos simultaneos, Numero de perfis, Outras descricoes}

    \createrelation{{\bf Conta Assina Plano}}{\primarykey{Nome Plano} (fk), \primarykey{Nome de usuario Conta} (fk)}

    \createrelation{{\bf Assinatura}}{\primarykey{Data vigor}, \primarykey{Nome Plano} (fk), \primarykey{Nome de usuario Conta} (fk)}

    \createrelation{{\bf Pagamento}}{\primarykey{Codigo da fatura}, \nullatt{Data vigor Assinatura} (fk), \nullatt{Nome Assinatura} (fk), \nullatt{Nome de usuario Assinatura} (fk), Valor, Periodo}

    \createrelation{{\bf Conta}}{\primarykey{Nome de usuario}, \nullatt{Forma de Pagamento} (fk), Senha, Nome, CPF, Email, Data de nascimento}

    \createrelation{{\bf Perfil Prefere Genero}}{\primarykey{Alias Perfil} (fk), \primarykey{Nome de usuario Perfil} (fk), \primarykey{Nome Genero} (fk), Nota}

    \createrelation{{\bf Genero}}{\primarykey{Nome}}

    \createrelation{{\bf Perfil Solicita Amizade}}{\primarykey{Alias fez solicitacao Perfil} (fk), \primarykey{Nome de usuario fez solicitacao Perfil} (fk), \primarykey{Alias recebe solicitacao Perfil} (fk), \primarykey{Nome de usuario recebe solicitacao Perfil} (fk), Aceitou, Data solicitacao}

    \createrelation{{\bf Perfil Assiste Midia}}{\primarykey{Alias Perfil} (fk), \primarykey{Nome usuario Perfil} (fk), \primarykey{Titulo Midia} (fk), Id Review (fk), Tempo}

    \createrelation{{\bf Publica}}{\primarykey{Id}, Nota, Data nota, Comentario, Data comentario}

    \createrelation{{\bf Gerencia}}{\nullatt{Alias adulto gerencia Perfil} (fk), \nullatt{Nome usuario adulto gerencia Perfil} (fk), \primarykey{Alias kids gerenciado Perfil} (fk), \primarykey{Nome usuario kids gerenciado Perfil} (fk)}

    \createrelation{{\bf Audio Midia}}{\primarykey{Audio}, \primarykey{Titulo}(fk Midia)}

    \createrelation{{\bf Legenda Midia}}{\primarykey{Legenda}, \primarykey{Titulo}(fk Midia)}

    \createrelation{{\bf Midia Pertence Genero}}{\primarykey{Titulo Midia} (fk), \primarykey{Nome Genero} (fk)}
    
    \createrelation{{\bf Midia Pertence Temporada}}{\primarykey{Titulo Midia} (fk), \primarykey{Numero Temporada} (fk)}

    \createrelation{{\bf Temporada Midia}}{\primarykey{Titulo Midia} (fk), \primarykey{Numero Temporada} (fk)}

    \createrelation{{\bf Review}}{\primarykey{Id}, Nota, Data Nota, Comentario, Data Comentario}

    \createrelation{{\bf Perfil Assiste Midia}}{\primarykey{Alias Perfil} (fk), \primarykey{Nome de usuario Perfil} (fk), \primarykey{Titulo Midia} (fk), Id Review (fk), Tempo}

    \createrelation{{\bf Acesso}}{\primarykey{Timestamp}, \primarykey{Nome Dispositivo} (fk), \primarykey{Alias Perfil} (fk), Ip}

    \createrelation{{\bf Perfil Possui Dispositivo}}{\primarykey{Id Dispositivo} (fk), \primarykey{Alias Perfil} (fk), \primarykey{Nome de usuario Perfil} (fk)}

    \createrelation{{\bf Dispositivo}}{\primarykey{Id}, Nome, Sistema operacional}

    \createrelation{{\bf Amizade}}{\primarykey{Id}, \primarykey{Alias fez solicitacao Perfil} (fk), \primarykey{Nome de usuario fez solicitacao Perfil} (fk), \primarykey{Alias recebe solicitacao Perfil} (fk), \primarykey{Nome de usuario recebe solicitacao Perfil} (fk)}
    
    \createrelation{{\bf Amizade Recomenda Midia}}{\primarykey{Id Amizade} (fk), \primarykey{Alias fez solicitacao Perfil} (fk), \primarykey{Nome de usuario fez solicitacao Perfil} (fk), \primarykey{Alias recebe solicitacao Perfil} (fk), \primarykey{Nome de usuario recebe solicitacao Perfil} (fk), \primarykey{Titulo Midia} (fk), Comentario de recomendacao}

    % //NOTE:_todas as temporadas precisam ter uma série,_logo não teremos nenhum nulo_
    \createrelation{{\bf Temporada}}{\primarykey{Numero}, \nullatt{Nome Serie} (fk), Numero de episodios}

    \createrelation{{\bf Serie}}{\primarykey{Nome}, Numero temporadas, Classificacao}

    \createrelation{{\bf Pessoa}}{\primarykey{Nome}}

    % //NOTE:_Pessoa só participa da relação Participa caso a_Mídia seja filme
    \createrelation{{\bf Midia Pessoa}}{\primarykey{Titulo Midia} (fk), \primarykey{Nome Pessoa} (fk), Diretor, Ator}

    \createrelation{{\bf Serie Pessoa}}{\primarykey{Nome Serie} (fk), \primarykey{Nome Pessoa} (fk), Diretor, Ator}
    
    \createrelation{{\bf Pessoa Participa Midia}}{\primarykey{Titulo Midia} (fk), \primarykey{Nome Pessoa} (fk), Diretor, Ator}
    
    \createrelation{{\bf Pessoa Participa Serie}}{\primarykey{Titulo Midia} (fk), \primarykey{Nome Pessoa} (fk), Diretor, Ator}

\newpage
\subsection{Dependências Funcionais}
    
    \singlespacing
    {\bf Tabela Plano: }
    \begin{itemize}
        \item \{Nome\} \rarrow \{Preço, Qualidade, Videos simultaneos, Numero de perfis, Outras descricoes\} \\
    \end{itemize}
    
    {\bf Tabela Conta Assina Plano: }
    \begin{itemize}
        \item \{Nome Plano, Nome de usuario Conta\} \rarrow \{\} \\
    \end{itemize}
    
    {\bf Tabela Assinatura: }
    \begin{itemize}
        \item \{Data vigor, Nome Plano, Nome de usuario Conta\} \rarrow \{\} \\
    \end{itemize}
    
    {\bf Tabela Pagamento: }
    \begin{itemize}
        \item \{Codigo da fatura\} \rarrow \{Valor, Periodo, Data vigor Assinatura, Nome Assinatura, Nome de usuario Assinatura\}\\
    \end{itemize}
    
    {\bf Tabela Forma de Pagamento: }
    \begin{itemize}
        \item \{Metodo\} \rarrow \{\} \\
    \end{itemize}
    
    {\bf Tabela Cartao de Credito: }
    \begin{itemize}
        \item \{Forma de Pagamento\} \rarrow \{Numero do Cartao, Bandeira, Codigo de seguranca, Vencimento do cartao, Nome\}
        \item \{Numero do Cartao\} \rarrow \{Bandeira, Codigo de seguranca, Vencimento do cartao, Nome\} \\
    \end{itemize}
    
    {\bf Tabela Deposito Bancario: }
    \begin{itemize}
        \item \{Forma de Pagamento\} \rarrow \{Conta, Agencia, Banco, Nome titular, Sobrenome titular, CPF\}
        \item \{Conta, Agencia\} \rarrow \{Banco, Nome titular, Sobrenome titular, CPF\}\\
    \end{itemize}
    
    {\bf Tabela Paypal: }
    \begin{itemize}
        \item \{Forma de Pagamento\} \rarrow \{Senha, Email paypal\} \\
    \end{itemize}
    
    {\bf Tabela Conta: }
    \begin{itemize}
        \item \{Nome de usuario\} \rarrow \{Forma de Pagamento, Senha, Nome, CPF, Email, Data de nascimento\} \\
    \end{itemize}
    
    {\bf Tabela Perfil Prefere Genero: }
    \begin{itemize}
        \item \{Alias Perfil, Nome de usuario Perfil, Nome Genero\} \rarrow \{Nota\} \\
    \end{itemize}
    
    {\bf Tabela Genero: }
    \begin{itemize}
        \item \{nome\} \rarrow \{\} \\
    \end{itemize}
    
    {\bf Tabela Perfil Solicita Amizade: }
    \begin{itemize}
        \item \{Alias fez solicitacao Perfil, Nome de usuario fez solicitacao Perfil, Alias recebe solicitacao Perfil, Nome de usuario recebe solicitacao Perfil\} \rarrow \{Aceitou, Data solicitacao\} \\
    \end{itemize}
    
    {\bf Tabela Perfil: }
    \begin{itemize}
        \item \{Alias, Nome de usuario Conta\} \rarrow \{Idade, Preferencia qualidade, Preferencia legenda, Preferencia idioma\}
        \item \{Idade\} \rarrow \{Faixa etaria\} \\
    \end{itemize}
    
    {\bf Tabela Perfil Assiste Midia: }
    \begin{itemize}
        \item \{Alias Perfil, Nome usuario Perfil, Titulo Midia\} \rarrow \{Id Review, Tempo\} \\
    \end{itemize}
    
    {\bf Tabela Review:}
    \begin{itemize}
        \item \{Id\} \rarrow \{Nota, Data nota, Comentario, Data comentario\} \\
    \end{itemize}
    
    {\bf Tabela Gerencia: }
    \begin{itemize}
        \item \{Alias kids gerenciado Perfil, Nome usuario kids gerenciado Perfil\} \rarrow \{Alias adulto gerencia Perfil, Nome usuario adulto gerencia Perfil\} \\
    \end{itemize}
    
    {\bf Tabela Midia: }
    \begin{itemize}
        \item \{Titulo\} \rarrow \{Tipo, Thumbnail, Lancamento, Duracao, Sinopse, Classificacao\} \\
    \end{itemize}
    
    {\bf Tabela Audio Midia: }
    \begin{itemize}
        \item \{Audio, Titulo\} \rarrow \{\} \\
    \end{itemize}
    
    {\bf Tabela Legenda Midia: }
    \begin{itemize}
        \item \{Legenda, Titulo\} \rarrow \{\} \\
    \end{itemize}
    
    {\bf Tabela Midia Pertence Genero: }
    \begin{itemize}
        \item \{Titulo Midia, Nome Genero\} \rarrow \{\} \\
    \end{itemize}
    
    {\bf Tabela Perfil Prefere Genero: }
    \begin{itemize}
        \item \{Alias Perfil, Nome de usuario Perfil, Nome Genero\} \rarrow \{Nota\} \\
    \end{itemize}
    
    {\bf Tabela Midia Pertence Temporada: }
    \begin{itemize}
        \item \{Titulo Midia, Numero Temporada\} \rarrow \{\} \\
    \end{itemize}
    
    {\bf Tabela Acesso: }
    \begin{itemize}
        \item \{Timestamp, Nome Dispositivo, Alias Perfil\} \rarrow \{Ip\} \\
    \end{itemize}
    
    {\bf Tabela Perfil Possui Dispositivo: }
    \begin{itemize}
        \item \{Id Dispositivo, Alias Perfil, Nome de usuario Perfil\} \rarrow \{\} \\
    \end{itemize}
    
    {\bf Tabela Dispositivo: }
    \begin{itemize}
        \item \{Id\} \rarrow \{Nome, Sistema operacional\} \\
    \end{itemize}
    
    {\bf Tabela Amizade: }
    \begin{itemize}
        \item \{Id, Alias fez solicitacao Perfil, Nome de usuario fez solicitacao Perfil, Alias recebe solicitacao Perfil, Nome de usuario recebe solicitacao Perfil\} \rarrow \{\} \\
    \end{itemize}
    
    {\bf Tabela Amizade Recomenda Midia: }
    \begin{itemize}
        \item \{Id Amizade, Alias fez solicitacao Perfil, Nome de usuario fez solicitacao Perfil, Alias recebe solicitacao Perfil, Nome de usuario recebe solicitacao Perfil, Titulo Midia\} \rarrow \{Comentario de recomendacao\} \\
    \end{itemize}
    
    {\bf Tabela Temporada: }
    \begin{itemize}
        \item \{Numero, Nome Serie\} \rarrow \{Numero de episodios\} \\
    \end{itemize}
    
    {\bf Tabela Serie: }
    \begin{itemize}
        \item \{Nome\} \rarrow \{Numero temporadas, Classificacao\} \\
    \end{itemize}
    
    {\bf Tabela Pessoa: }
    \begin{itemize}
        \item \{Nome\} \rarrow \{\} \\
    \end{itemize}
    
    {\bf Tabela Pessoa Participa Midia: }
    \begin{itemize}
        \item \{Titulo Midia, Nome Pessoa\} \rarrow \{Diretor, Ator\} \\
    \end{itemize}
    
    {\bf Tabela Pessoa Participa Serie: }
    \begin{itemize}
        \item \{Titulo Midia, Nome Pessoa\} \rarrow \{Diretor, Ator\} \\
    \end{itemize}

\subsection{Primeira Forma Normal}

    Não houve necessidade de alterar nada no esquema de relacionamentos pois após o mapeamento, elas já estavam na primeira forma normal.
    
\subsection{Segunda Forma Normal}

    Não houve necessidade de alterar nada no esquema de relacionamentos pois após o mapeamento, elas já estavam na segunda forma normal.
    
\subsection{Terceira Forma Normal}

    Houve necessidade de alterar a tabela Perfil. Nela, o atributo faixa etária era determinado pelo atributo idade. Por isso, também foi separado em duas tabelas: "Perfil" e "Idade Faixa Etaria".\\

\createrelation{{\bf Perfil}}{\primarykey{Alias}, \primarykey{Nome usuario Conta} (fk), \nullatt{Idade Faixa Etaria}(fk), Preferencia qualidade, Preferencia legenda, Preferencia idioma}\\

\createrelation{{\bf Idade Faixa Etaria}}{\primarykey{Idade}, Faixa Etaria}

\subsection{Forma de Boyce-Codd}

	
    Houve necessidade de alterar duas tabelas para que elas se ajustassem à BCNF: "Depósito Bancário" e "Cartão de crédito".\\
    Na tabela Depósito Bancário, os atributos Banco, Nome titular, Sobrenome titular e CPF estavam sendo determinados indiretamente pela Forma de pagamento e, portanto, a tabela foi separada em duas: "Depósito Bancário" e "Conta do Depósito Bancário".\\
    
    \createrelation{{\bf Deposito Bancario}}{\primarykey{Forma de Pagamento} (fk), Conta Deposito Bancario (fk), Agencia Deposito Bancario (fk)}\\

    \createrelation{{\bf Conta do Deposito Bancario}}{\primarykey{Conta, Agencia}, Banco, Nome titular, Sobrenome titular, CPF}\\
   
   Na tabela "Cartão de Crédito", o atributo não chave "Número do cartão" estava determinando os demais "Bandeira", "Codigo de seguranca", "Vencimento do cartao", "Nome".\\ 
   
        \createrelation{{\bf Cartao de Credito}}{\primarykey{Forma de Pagamento} (fk), Numero do cartao (fk)}\\

        \createrelation{{\bf Número do Cartao de Credito}}{\primarykey{Numero do cartao}, Bandeira, Codigo de seguranca, Vencimento do cartao, Nome}


\subsection{Quarta Forma Normal}

    Não houve necessidade de alterar nada no esquema de relacionamentos pois após o mapeamento, elas já estavam na quarta forma normal.
    

\end{document}

